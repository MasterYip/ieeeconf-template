%%%%%%%%%%%%%%%%%%%%%%%%%%%%%%%%%%%%%%%%%%%%%%%%%%%%%%%%%%%%%%%%%%%%%%%%%%%%%%%%
%2345678901234567890123456789012345678901234567890123456789012345678901234567890
%        1         2         3         4         5         6         7         8

\documentclass[letterpaper, 10 pt, conference]{ieeeconf}  % Comment this line out if you need a4paper

%\documentclass[a4paper, 10pt, conference]{ieeeconf}      % Use this line for a4 paper

\IEEEoverridecommandlockouts                              % This command is only needed if 
                                                          % you want to use the \thanks command

\overrideIEEEmargins                                      % Needed to meet printer requirements.

%In case you encounter the following error:
%Error 1010 The PDF file may be corrupt (unable to open PDF file) OR
%Error 1000 An error occurred while parsing a contents stream. Unable to analyze the PDF file.
%This is a known problem with pdfLaTeX conversion filter. The file cannot be opened with acrobat reader
%Please use one of the alternatives below to circumvent this error by uncommenting one or the other
%\pdfobjcompresslevel=0
%\pdfminorversion=4

% See the \addtolength command later in the file to balance the column lengths
% on the last page of the document

% Replace `english' with e.g. `spanish' to change the document language
\usepackage[english]{babel}

% Review revised text
\usepackage{soul} % Text highlighting
\usepackage{marginnote} % Margin notes
% Define color commands
\newcommand{\revise}[2]{
    \sethlcolor{yellow}\hl{#2} % Highlight revised text
    \marginnote{\color{red}\small C#1} % Margin note with comment number
}

% Review Box
\usepackage{xcolor}
\usepackage{tcolorbox}
\tcbuselibrary{breakable} % Support for spanning columns/pages
% Define review comment annotation style
\newcounter{reviewcounter} % Auto-numbering counter
\newtcolorbox{revisebox}[2][]{
    breakable, % Support spanning columns
    colback=blue!5!white, % Background color
    colframe=blue!50!black, % Border color
    title={\textbf{Revisions to Comment~#2}}, % Title (comment number)
    fonttitle=\small,
    left=0mm,right=0mm,top=1mm,bottom=1mm, % Margins
    #1 % Custom parameters
}

% Useful packages
\usepackage{cite}
% \usepackage{subfig}
\usepackage{graphicx}
\usepackage{subcaption}
\usepackage{multirow}
\usepackage{booktabs}
\usepackage[colorlinks=true, allcolors=blue, hyperfootnotes=true]{hyperref}
% \usepackage{tabularx} %This package must be placed after package {hyperref}, otherwise footnote marks are NOT treated as hyperlinks.
\usepackage{utils/symbols}
% \usepackage{tikz}
% \usepackage{amssymb}
\usepackage{amsmath}
% \usepackage{algorithm}
% \usepackage{algorithmicx}
% \usepackage{algpseudocode}
% \usepackage{mathtools}
% \usepackage{caption}
% \usepackage{placeins}


\title{\LARGE \bf
ROS4VSN: Visual Semantic Navigation in Real Robots with ROS}


\author{C. Guti\'errez-\'Alvarez$^{1*}$, P. R\'ios-Navarro$^{1*}$, R. Flor-Rodr\'iguez-Rabad\'an$^{1}$, \\F. J. Acevedo-Rodr\'iguez$^{1}$ and R. J. L\'opez-Sastre$^{1}$
    \thanks{*Equal contribution.}% <-this % stops a space
    \thanks{$^{1}$University of Alcal\'a, Department of Signal Theory and Communications, Spain. Email: \texttt{\footnotesize carlos.gutierrezalva@uah.es}}
    % \thanks{$^2$Code will be released.}
    % \thanks{$^3$ Link: \url{https://youtu.be/s1rem3c6fw8}.}
}

\begin{document}



\maketitle
\thispagestyle{empty}
\pagestyle{empty}


%%%%%%%%%%%%%%%%%%%%%%%%%%%%%%%%%%%%%%%%%%%%%%%%%%%%%%%%%%%%%%%%%%%%%%%%%%%%%%%%
\begin{abstract}

    This electronic document is a live template.
    The various components of your paper [title, text, heads, etc.] are already defined on the style sheet, as illustrated by the portions given in this document.

\end{abstract}


%%%%%%%%%%%%%%%%%%%%%%%%%%%%%%%%%%%%%%%%%%%%%%%%%%%%%%%%%%%%%%%%%%%%%%%%%%%%%%%%

\section{INTRODUCTION}

Hello Test introduction~ \cite{Sethian1996}.

Therefore, \revise{3.1}{simple gradient ascent search for actor updates can be performed.}
In DDPG, the policy to be learned is deterministic, and it might not try
enough to explore the environment. To increase exploration ability,
zero-mean random noise can be added to deterministic actions and initial
states can be sampled from a given state distribution at the beginning
of environment interaction. The noise is added to the action before it
is executed in the environment. The noise is generated by a Gaussian process,
which is a stochastic process that can be used to model the uncertainty in the environment.
The Gaussian process is defined by its mean and covariance functions,
which determine the shape of the noise distribution. The mean function is typically set to zero,
and the covariance function is chosen to
be a radial basis function (RBF) kernel, which allows for smooth variations in the noise.

\begin{revisebox}{3.2} % {1}为审稿意见编号
    Before: The accuracy is 90\%. \\
    \revise{3.2}{% 标红修改内容
        After: The accuracy is 95\% with additional data augmentation.}
    Therefore, simple gradient ascent search for actor updates can be performed.
    In DDPG, the policy to be learned is deterministic, and it might not try
    enough to explore the environment. To increase exploration ability,
    zero-mean random noise can be added to deterministic actions and initial
    states can be sampled from a given state distribution at the beginning
    of environment interaction. The noise is added to the action before it
    is executed in the environment. The noise is generated by a Gaussian process,
    which is a stochastic process that can be used to model the uncertainty in the environment.
    The Gaussian process is defined by its mean and covariance functions,
    which determine the shape of the noise distribution. The mean function is typically set to zero,
    and the covariance function is chosen to
    be a radial basis function (RBF) kernel, which allows for smooth variations in the noise.
    \begin{gather}
        % Maxwell equations
        \nabla \cdot \mathbf{E} = \frac{\rho}{\epsilon_0} \\
        \nabla \cdot \mathbf{B} = 0 \\
        \nabla \times \mathbf{E} = -\frac{\partial \mathbf{B}}{\partial t} \\
        \nabla \times \mathbf{B} = \mu_0 \mathbf{J} + \mu_0 \epsilon_0 \frac{\partial \mathbf{E}}{\partial t}
    \end{gather}
\end{revisebox}




\begin{table}[t]
    \centering
    \caption{Example of good table}
    \label{tab:vlv}
    \begin{tabular}{c|cccc}
        \toprule
        \textit{\textbf{Object Goal}} & \textit{\textbf{Successful episodes}} & \textit{\textbf{SR}} & \textit{\textbf{Avg. number of actions}} \\ \midrule
        Chair                         & 6/15                                  & 40\%                 & 30                                       \\
        Sofa                          & 6/15                                  & 40\%                 & 65                                       \\
        Table                         & 6/15                                  & 40\%                 & 42                                       \\
        Bed                           & 3/15                                  & 20\%                 & 39                                       \\
        Toilet                        & 1/15                                  & 6,67\%               & 42                                       \\ \bottomrule
    \end{tabular}
\end{table}


\bibliographystyle{IEEEtran}
\bibliography{library}




\end{document}
